\section{Background}
\label{sec:background}

\subsection{What is Tor and How Does It Work}
Known as the second-generation onion router, Tor is an anonymous, circuit-based, low-latency communication service designed to anonymize TCP-based applications like web browsing, secure shell, and instant messaging~\cite{dingledine2004tor}. Tor does this by obfuscating the path that data packets take to get from the source IP address to the destination IP address and by ensuring that no observer at a single point along the communication channel can directly identify both the source and destination addresses. In order to connect and send data to a server, a Tor client obtains a list of Tor nodes (or "onion routers") from a directory server. Then, using this list of Tor nodes, the client picks a randomly generated 3-tuple path to the destination server, building a circuit. The circuit is comprised of an entry or guard relay, a middle relay, and an exit relay. Each relay in the path knows only its predecessor and successor, but no other nodes in the circuit. Before sending data to the server, the client negotiates a separate set of symmetric sessions keys for each hop along the circuit. Traffic flows down the circuit in fixed-size cells, which are unwrapped by a symmetric key at each node (like the layers of an onion) and relayed downstream~\cite{dingledine2004tor}. All communication links between the nodes of the circuit are encrypted except for the traffic leaving the exit relay to the destination IP address. 

Relays are operated by volunteers around the world, and their IP addresses are public and known. Additionally, each relay advertises how much bandwidth it can handle so that Tor users' traffic can be load-balanced.

\subsection{RSA public-key cryptosystem}
Because Tor relies on TLS and asymmetric key cryptography to encrypt messages that travel between the nodes of the Tor relay circuits, in this paper, I focus on RSA as used in the TLS protocol. RSA public-key cryptosystems are comprised of two keys: a publicly reported encryption key and a privately held decryption key~\cite{rivest1978method}. The encryption key, or RSA public key, is comprised of a pair of positive integers, an exponent $e$ and a modulus $N$. The modulus $N$ is the product of two very large random prime numbers $p$ and $q$. The corresponding decryption key, or RSA private key, is comprised of the positive integer pair $d$ and $N$ where $d = e^{-1} mod (p - 1)(q - 1)$. The decryption exponent $d$ is efficient to compute if the factorization of $N$ is known. Thus, the strength of the RSA public-key cryptosystem depends on the enormous difficulty of factoring the modulus $N$ into its factors $p$ and $q$ and relies on the randomness of $p$ and $q$ to ensure that the modulus $N$ is not shared among multiple public keys.

\subsection{Factorable RSA keys}
As with all things, no cryptographic system is 100\% foolproof, for various reasons. One such reason is the existence of weak keys, which can be exploited to break cryptographic transport protocols like TLS and SSH much more efficiently. In this paper, I focus on one type of weak key vulnerability: factorable RSA keys, meaning pairs of RSA key moduli that can efficiently be factored into their $p$s and $q$s as a result of sharing a single common prime factor. Unlike attempting to factor a well-generated 1024-bit RSA modulus, a computationally infeasible task, computing the greatest common divisor (GCD) of two moduli in order to detect if they share a prime factor can be performed in microseconds. For example, say you have two distinct RSA moduli $N_1 = pq_1$ and $N_2 = pq_2$ that share the prime factor p. An attacker could quickly and easily compute the GCD of $N_1$ and $N_2$ in order to obtain $p$, then divide the moduli by $p$ to find $q_1$ and $q_2$, and finally use the equation in the previous section to compute both private keys.

\subsection{Detection of widespread weak keys in network devices}
In an Internet-wide survey of public keys,~\cite{heninger2012mining} computed the pair-wise GCD of 11,170,883 distinct RSA moduli yielding 2,314 distinct prime divisors, which divided 16,717 distinct public keys. From this, the authors were able to obtain private keys for 0.40\% of the TLS certificates in their scan data. ~\cite{heninger2012mining} attributed many of these weak factorable RSA moduli to entropy problems in flawed implementations of random number generators. The authors of~\cite{heninger2012mining} found headless, embedded, and low-resource network devices to be particularly plagued by this issue.~\cite{heninger2012mining}'s analysis revealed that in some resource-constrained network devices, the operating system random number generator may not incorporate external sources of entropy when it is used by an application to generate a cryptographic key. As a result, during the key-generation process, two applications running on different systems may have identical entropy pool states during generation of the first prime factor of the RSA modulus, creating identical prime factors~\cite{hastings2016weak}. Furthermore, in an experiment they conducted,~\cite{heninger2012mining} discovered that OpenSSL could generate factorable keys due to a boot-time "entropy hole" vulnerability in the Linux random number generator, causing its /dev/urandom output to be deterministic. 

~\cite{heninger2012mining} findings reiterated the importance of randomness in modern cryptography, where the strength and security of the cryptographic system often depends on keys being chosen uniformly at random.

\subsection{Implication of weak Tor keys}
While many Tor relays run on large PCs and VPS data centers--which are generally not as susceptible to problems related to inadequate randomness due to their greater ability to collect sufficient entropy--it has been noted that some of the relays run on smaller devices, such as Raspberry Pi products. These and other potentially resource-constrained devices might not have enough sources of entropy, causing them to generate weak keys with shared prime factors. The implications of factorable RSA signing keys in Tor relay servers are serious. As discussed earlier, discovering that a relay's RSA key is factorable means that an attacker can compromise the key pair by gaining access to the private key. This makes it possible to do a number of things. First, the attacker can launch person-in-the-middle attacks. Let us assume that Becky is trying to deanonymize Trayvon. If Trayvon's Tor client connects to his guard relay in Sweden,
and Becky managed to factor this relay's key, she can now decrypt the network traffic Trayvon is sending to his guard relay. However, given Tor's layered encryption, Becky is still left with two layers of encryption to break--the middle and the exit relay.

Second, every Tor relay has an important long-term RSA key called the relay's "identity key," which essentially defines the identity of the relay. If an adversary steals this key, then she can impersonate the relay. The main use of this identity key in Tor is to sign the relay's "descriptor." This document includes various information about the relay (e.g., its IP address, some keys, contact information, etc.) and is signed with its identity key. Since the identity key of every relay is included in the Tor consensus, all clients know the identity keys of all relays. The consensus acts as the public key infrastructure (PKI) of Tor. When a client fetches a Tor descriptor, it checks the consensus for the right identity key, and then verifies the descriptor signature. This ensures that the descriptor was created by an entity that knew the private
identity key, and hence is not a fake descriptor created by a third party. If an attacker were to break the identity key of a relay (by deriving the private key from its public key), she could start signing descriptors in its name, and publishing them. This way the adversary could publish whatever information she wanted in the descriptor (e.g., her own IP address, own keys, etc.), and Tor clients would be fooled into thinking it's the actual relay, but instead, would be connecting to the attacker. Thus, one can imagine an exploit that could hack most identify keys of Tor relays. From this vantage point, the attacker could now redirect all clients to her own machines and monitor their traffic. 

Effectively, factoring an RSA key is equivalent to controlling the compromised Tor relay. While it doesn't immediately allow an adversary to link a Tor user's IP address to her destination, it is a critical step that can bring the attacker much closer to that goal.
