\section{Conclusions}
\label{sec:conclusion}

In order to detect potentially weak factorable keys in Tor, I used~\cite{heninger2012mining}'s fastgcd solution for efficiently computing the greatest common divisor for every pair of integers in a large set. While it is not computationally practical to factor large integer RSA keys, it is computationally practical to compute their pair-wise GCDs. Thus, if an attacker is able to find a pair of RSA key moduli with shared prime factors, then she can easily use this information to compute the private keys~\cite{heninger2012mining, lenstra2012ron}. Knowing the private key of a Tor relay server has serious security implications, and is a critical first step to compromising the anonymity of the system. With private key in hand, that adversary now has control of the relay. In a survey of all 3,763,821 Tor relay server RSA public keys, I found that 0.60\% of all long-term signing keys had moduli with shared prime factors and found 10 long-term signing keys with repeated moduli. I will continue my quest to protect the anonymity of Tor users by extending my work on this project.