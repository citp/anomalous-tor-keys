\section{Conclusion}
\label{sec:conclusion}

Previous research has studied the problem of weak RSA keys in different 
systems, and we wondered if there might be weak keys in the Tor 
network, too, that have the potential to compromise Tor users' safety.  
Thus, the goal of our work was to look for weak and anomalous keys in Tor 
and investigate their origins in order to address the problem.  We 
achieved our goal by gathering all the archived RSA keys used in Tor since 
2005 and examining them for common prime factors as previous work had done.  
Additionally, we looked for keys that shared the same moduli and 
for keys that had non-standard public exponents.

We found that Tor researchers inadvertently created weak keys while 
conducting experiments on Tor.  We also found indications that other 
entities had purposely created weak and anomalous keys in order to 
attack Tor's onion services.  

Our work demonstrates that the presence of weak and anomalous RSA keys in 
Tor is definitely a sign of something questionable going on that should 
be paid attention to.  Indeed our findings have already been used to 
extend Tor safety scripts to look for non-standard RSA exponents, 
which go against Tor's specification, and malicious relays have been busted.  
It's yet another demonstration of ``trust but verify'' at work.
