\begin{abstract}
{
In its more than ten years of existence, the Tor network has seen hundreds of
thousands of relays come and go.  Each relay maintains several RSA keys,
amounting to millions of keys, all archived by The Tor Project.
%
In this paper, we analyze 3.7 million RSA public keys of Tor relays.  We \first
check if any relays share prime factors or moduli, \second identify relays that
use non-standard exponents, and \third characterize malicious relays that we
discovered in the first two steps.
%
Our experiments revealed that ten relays shared moduli, and 3,577 relays---most
part of a research project---shared prime factors, allowing adversaries to
reconstruct private keys.  We further discovered 122 relays that used
non-standard RSA exponents, presumably in an attempt to attack onion services.
By simulating how onion services are positioned in Tor's distributed hash table,
we could identify three onion services that were likely targeted by these
malicious relays.
}
\end{abstract}
