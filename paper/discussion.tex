\section{Discussion}
\label{sec:discussion}
\paragraph{Implications of weak Tor keys}
As touched on earlier in Section~\ref{sec:tor-network}, the main use of the
identity key in Tor is to sign the relay's descriptor, which includes various
information about the relay, \eg, its IP address, contact information, etc.
Since the identity key of every relay is included in their descriptors, all
clients know the identity keys of all relays. The consensus acts as the public
key infrastructure (PKI) of Tor. When a client fetches a Tor descriptor, it
checks the consensus for the right identity key, and then verifies the
descriptor signature. This ensures that the descriptor was created by an entity
that knew the private identity key, and hence is not a fake descriptor created
by a third party. If an attacker were to break the identity key of a relay (as
we demonstrated), she could start signing descriptors in the relay's name and
publishing them. The adversary could publish whatever information she wanted in
the descriptor \eg her own IP address, keys, etc. Tor clients would be fooled
into thinking it's the actual relay, but instead, would be connecting to the
attacker. Thus, one can imagine an exploit that could hack most identify keys of
Tor relays, allowing the attacker to redirect all clients to her own machines
and monitor their traffic.

\paragraph{Preventing non-standard exponents}
Recall that the Tor reference implementation hard-codes its public RSA exponent
to 65,537~\cite[\S~0.3]{torspec}.  The Tor Project could prevent non-standard
exponents by having the directory authorities reject relays whose descriptors
have an RSA exponent other than 65,537, thus slowing down the search for
fingerprint prefixes.  Adversaries would then have to iterate over the primes
$p$ or $q$ instead of the exponent, rendering the search more computationally
expensive.  Given that we discovered only 122 unusual exponents in over ten
years of data, we believe that rejecting non-standard exponents is a viable
defense in depth.

\paragraph{Analyzing onion service public keys}
Future work should shed light on the public keys of onion services.  Onion
services have an incentive to modify their fingerprints to make them both
recognizable and easier to remember.  Facebook, for example, was lucky to
obtain the easy-to-remember onion domain
\url{facebookcorewwwi.onion}~\cite{facebook}.  The tool Scallion assists onion
service operators in creating such vanity domains~\cite{scallion}.  The
implications of vanity domains on usability and security are still poorly
understood~\cite{vanity-domains}.  Unlike the public keys of relays, onion
service keys are not archived, so a study would have to begin with actively
fetching onion service keys.

\paragraph{\textit{In vivo} Tor research}
Caution must be taken when conducting research using the live Tor network.
Section~\ref{sec:shared-primes} showed how a small mistake in key generation led
to many vulnerable Tor relays.  To keep its users safe, The Tor Project has
recently launched a research safety board whose aim is to assist researchers in
safely conducting Tor measurement studies~\cite{safety-board}.  This may entail
running experiments in private Tor networks that are controlled by the
researchers, or using network simulators such as Shadow~\cite{stem}.
