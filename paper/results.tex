\section{Results}
\label{sec:results}
We present our results in four parts, starting with shared prime factors
(Section~\ref{sec:shared-primes}), followed by shared moduli
(Section~\ref{sec:shared-moduli}), unusual exponents 
(Section~\ref{sec:unusual-exponents}), and finally, targeted onion services
(Section~\ref{sec:targeted-onion-services}).

\subsection{Shared prime factors}
\label{sec:shared-primes}
Among all 588,945 identity keys, fastgcd found that 3,557 (0.6\%)
moduli share prime factors.  We believe that 3,555 of these keys were all
controlled by a single research group, and upon contacting the authors of the
Security \& Privacy 2013 paper entitled ``Trawling for Tor hidden
services''~\cite{Biryukov2013a}, we received confirmation that these relays
were indeed run by their research group.  The authors informed us that the weak
keys were caused by a shortcoming in their key generation tool. The issue
stemmed from the fact that their tool first generated thousands of prime numbers
and then computed multiple moduli using combinations of those prime numbers in a
greedy fashion without ensuring that the same primes were not reused.  Because
of the following shared properties, we are confident that all relays were
operated by the researchers:

\begin{enumerate}
	\item All relays were online either between November 11, 2012 and
		November 16, 2012 or between January 14, 2013 and February 6, 2013,
		suggesting two separate experiments. We verified this by checking how
		long the relays stayed in the Tor network consensus. The Tor consensus
		is updated hourly and documents which relays are available at a
		particular time. This data is archived by The Tor Project and is made
		publicly available on the CollecTor platform~\cite{collector}.

	\item All relays exhibited a predictable port assignment scheme.  In
		particular, we observed ports \{7003, 7007, \dots, 7043, 7047\} and
		\{8003, 8007, \dots, 8043, 8047\}.

	\item Except for two machines that were located in Russia and Luxembourg,
		all machines were hosted in Amazon's EC2 address space.  All machines
		except the one located in Luxembourg used Tor version 0.2.2.37.

	\item All physical machines had multiple fingerprints.  1,321 of these 3,557
		relays were previously characterized by Winter
		\ea~\cite[\S~5.1]{Winter2016a}.
\end{enumerate}

The remaining two keys belonged to a relay named ``Desaster\-Blaster,'' whose
origins we could not determine. Its router descriptor indicates that the relay
has been hosted on a MIPS machine which might suggest an embedded device with a
weak random number generator:

{\footnotesize
\begin{verbatim}
router DesasterBlaster 62.226.55.122 9001 0 0
platform Tor 0.2.2.13-alpha on Linux mips
\end{verbatim}
}

To further investigate, we checked whether the relay ``Desaster\-Blaster'' shares
prime factors with any other relays. It appears that the relay has rotated
multiple identity keys, and it only shares prime factors with its own keys.
Unfortunately the relay did not have any contact information configured which is
why we could not get in touch with its operator.


\subsection{Shared moduli}
\label{sec:shared-moduli}
In addition to finding shared prime factors, we discovered relays that share a
\emph{modulus}, giving them the ability to calculate each other's private keys.
With $p$, $q$, and each other's $e$s in hand, the two parties can compute
each other's decryption exponent $d$, at which point both parties now know the
private decryption keys.

Table~\ref{tab:moduli} shows these ten relays with shared moduli clustered into
four groups. The table shows the relays' truncated, four-byte fingerprint, IP
addresses, and RSA exponents.  Note that the Tor client hard-codes the RSA
exponent to 65,537~\cite[\S~0.3]{torspec}, a recommended value that is resistant
to attacks against low public exponents~\cite[\S~4]{Boneh1999a}.  Any value
other than 65,537 indicates non-standard key generation.  All IP addresses were
hosted by OVH, a popular French hosting provider, and some of the IP addresses
hosted two relays, as our color coding indicates.  Finally, each group shared a
four- or five-digit prefix in their fingerprints.  We believe that a single
attacker controlled all these relays with the intention to manipulate the
distributed hash table that powers onion services~\cite{Biryukov2013a}---the
shared fingerprint prefix is an indication.  Because the modulus is identical,
we suspect that the attackers iterated over the relays' RSA exponents to come up
with the shared prefix.  The Tor Project informed us that it discovered and
blocked these relays in August 2014 when they first came online.

\begin{table}[t]
	\caption{Four groups of relays that have a shared modulus.  All relays
	further share a fingerprint prefix in groups of two or three, presumably to
	manipulate Tor's distributed hash table.}
	\label{tab:moduli}

	\centering
	\begin{tabular}{l l r}
	\toprule

	Short fingerprint & IP address & Exponent \\
	\midrule

	\texttt{\setlength{\fboxsep}{0pt}%
	\colorbox{highlight1}{\strut 838A}296A} & 188.165.164.163 &
	1,854,629 \\

	\texttt{\setlength{\fboxsep}{0pt}%
	\colorbox{highlight1}{\strut 838A}305F} &
	{\setlength{\fboxsep}{0pt}\colorbox{highlight3}{\strut 188.165.26.13}} &
	718,645 \\

	\texttt{\setlength{\fboxsep}{0pt}%
	\colorbox{highlight1}{\strut 838A}71E2} &
	{\setlength{\fboxsep}{0pt}\colorbox{highlight2}{\strut 178.32.143.175}} &
	220,955 \\

	\midrule

	\texttt{\setlength{\fboxsep}{0pt}%
	\colorbox{highlight1}{\strut 2249E}B42} &
	{\setlength{\fboxsep}{0pt}\colorbox{highlight3}{\strut 188.165.26.13}} &
	4,510,659 \\

	\texttt{\setlength{\fboxsep}{0pt}%
	\colorbox{highlight1}{\strut 2249E}C78} &
	{\setlength{\fboxsep}{0pt}\colorbox{highlight2}{\strut 178.32.143.175}} &
	1,074,365 \\

	\midrule

	\texttt{\setlength{\fboxsep}{0pt}%
	\colorbox{highlight1}{\strut E1EF}A388} & 188.165.3.63 &
	18,177 \\

	\texttt{\setlength{\fboxsep}{0pt}%
	\colorbox{highlight1}{\strut E1EF}8985} &
	{\setlength{\fboxsep}{0pt}\colorbox{highlight4}{\strut 188.165.138.181}} &
	546,019 \\

	\texttt{\setlength{\fboxsep}{0pt}%
	\colorbox{highlight1}{\strut E1EF}9EB8} &
	{\setlength{\fboxsep}{0pt}\colorbox{highlight5}{\strut 5.39.122.66}} &
	73,389 \\

	\midrule

	\texttt{\setlength{\fboxsep}{0pt}%
	\colorbox{highlight1}{\strut 410B}A17E} &
	{\setlength{\fboxsep}{0pt}\colorbox{highlight4}{\strut 188.165.138.181}} &
	1,979,465 \\

	\texttt{\setlength{\fboxsep}{0pt}%
	\colorbox{highlight1}{\strut 410B}B962} &
	{\setlength{\fboxsep}{0pt}\colorbox{highlight5}{\strut 5.39.122.66}} &
	341,785 \\

	\bottomrule
	\end{tabular}
\end{table}

\subsection{Unusual exponents}
\label{sec:unusual-exponents}
Having accidentally found a number of relays with non-standard exponents in
Section~\ref{sec:shared-moduli}, we checked if our dataset featured more relays
with exponents other than 65,537.  Non-standard exponents may indicate that a
relay was after a specific fingerprint in order to position itself in Tor's hash
ring.  To obtain a fingerprint with a given prefix, an adversary repeatedly has
to modify any of the underlying key material $p$, $q$, or $e$ until they result
in the desired prefix.  Repeated modification of $e$ is significantly more
efficient than modifying $p$ or $q$ because it is costly to verify if a large
number is prime.  Leveraging this method, the tool Scallion~\cite{scallion}
generates vanity onion service domains by iterating over the service's public
exponent.

Among all of our 3.7 million keys, 122 possessed an exponent other than 65,537.
One relay had both non-standard identity \emph{and} onion key exponents while
all remaining relays only had non-standard identity key exponents.  Ten of these
relays further had a shared modulus, which we discuss in
Section~\ref{sec:shared-moduli}.  Assuming that these relays positioned
themselves in the hash ring to attack an onion service, we wanted to find out
what onion services they targeted.  One can identify the victims by first
compiling a comprehensive list of onion services and then determining each
service's position in the hash ring at the time the malicious HSDirs were
online.

\subsection{Identifying targeted onion services}
\label{sec:targeted-onion-services}

We obtained a list of onion services by augmenting the list of the Ahmia
search engine~\cite{ahmia} with services that we discovered via Google searches
and by contacting researchers who have done similar work~\cite{Matic2015a}.  We
ended up with a list of 17,198 onion services that were online at some point in
time.  Next, we developed a tool that takes as input our list of onion services
and the malicious HSDirs we discovered.\footnote{Both the tool and our list of
onion services are available online at
\url{https://nymity.ch/anomalous-tor-keys/}.} The tool then calculates all
descriptors these onion services ever generated and checks if any HSDir shared
five or more hex digits in its fingerprint prefix with the onion service's
descriptor.  We chose the threshold of five manually because it is unlikely to
happen by chance yet easy to create a five-digit collision.

It is difficult to identify all targeted onion services because \first our list
of onion services does not tell us when a service was online, \second an HSDir
could be responsible for an onion service simply by chance rather than on
purpose, resulting in a false positive, and \third our list of onion services is
not exhaustive, so we are bound to miss victims.  Nevertheless our tool
identified four onion services (see Table~\ref{tab:targeted}) for which we have
strong evidence that they were purposely targeted.  While HSDirs are frequently
in the vicinity of an onion service's descriptor by accident, the probability of
being in its vicinity for several days in a row or cover both replicas by chance
is negligible.  Table~\ref{tab:collisions} shows all partial collisions in
detail.  Because none of these four services seem to have been intended for
private use, we are comfortable publishing them.

\begin{table}[t]
	\caption{The four onion services that were most likely targeted at some
	point.  The second column indicates if only one or both replicas were
	attacked while the third column shows the duration of the attack.}
	\label{tab:targeted}
	\centering
	\begin{tabular}{l r l}
	\toprule
	Onion service & Replicas & Attack duration \\
	\midrule
	\texttt{22u75kqyl666joi2.onion} & 2 & Two consecutive days \\
	\texttt{n3q7l52nfpm77vnf.onion} & 2 & Six non-consecutive days \\
	\texttt{silkroadvb5piz3r.onion} & 1 & Nine mostly consecutive days \\
	\texttt{thehub7gqe43miyc.onion} & 2 & One day \\
	\bottomrule
	\end{tabular}
\end{table}

\begin{description}
	\item[\texttt{22u75kqyl666joi2.onion}] The service appears to be offline
		today, so we were unable to see for ourselves what it hosted.  According
		to cached index pages we found online, the onion service used to host a
		technology-focused forum in Chinese.  A set of relays targeted the onion
		service on both August 14 and 15, 2015 by providing nine out of the
		total of twelve responsible HSDirs.

	\item[\texttt{n3q7l52nfpm77vnf.onion}] As of February 2017, the service is
		still online, hosting the ``Marxists Internet Archive,'' an online
		archive of literature.\footnote{The onion service seems to be identical
		to the website \url{https://www.marxists.org} (visited on 2017-05-09).}
		A set of relays targeted the onion service from November 27 to December
		4, 2016.  The malicious HSDirs acted inconsistently, occasionally
		targeting only one replica.

	\item[\texttt{silkroadvb5piz3r.onion}] The onion service used to host the
		Silk Road marketplace, whose predominant use was a market for narcotics.
		The service was targeted by a set of relays from May 21 to June 3, 2013.
		The HSDirs were part of a measurement experiment that resulted in a blog
		post~\cite{OCearbhaill2013a}.

	\item[\texttt{thehub7gqe43miyc.onion}] The onion service used to host a
		discussion forum, ``The Hub,'' focused on darknet markets.  A set of
		relays targeted both of The Hub's replicas from August 22, 2015.
\end{description}

Our data cannot provide insight into what the HSDirs did once they controlled
the replicas of the onion services they targeted.  The HSDirs could have counted
the number of client requests, refused to serve the onion service's descriptor
to take it offline, or correlate client requests with guard relay traffic in
order to deanonymize onion service visitors as it was done by the CMU/SEI
researchers in 2014~\cite{Dingledine2014a}.  Since these attacks were
short-lived we find it unlikely that they were meant to take offline the
respective onion services.
