\section{Results}
\label{sec:results}
We present our results in three parts, starting with shared prime factors in
Section~\ref{sec:shared-primes}, followed by shared moduli in
Section~\ref{sec:shared-moduli}, and ending with unusual exponents 
in Section~\ref{sec:unusual-exponents}.

\subsection{Shared prime factors}
\label{sec:shared-primes}
Among all 588,945 identity keys, fastgcd found that 3,577 (0.61\%) moduli share
prime factors.  We believe that 3,575 of these keys were all controlled by a
single research group.  Upon contacting the authors of the Security \& Privacy
2013 paper ``Trawling for Tor hidden services''~\cite{Biryukov2013a}, we
received confirmation that these relays indeed were run by the research group.
Because of the following shared properties, it was easy to determine that all
relays were operated by the same group.

\begin{enumerate}
	\item All relays were online either in between November 11, 2012 and
		November 16, 2012 or in between January 14, 2013 and February 6, 2013,
		suggesting two separate experiments.

	\item All relays had a predictable port assignment scheme.  In particular,
		we observed ports $\{7003, 7007, \dots, 7043, 7047\}$ and $\{8003, 8007,
		\dots, 8043, 8047\}$.

	\item Except two physical machines that were located in Russia and
		Luxembourg, all machines were hosted in Amazon's EC2 address space.  All
		machines except the one located in Luxembourg used Tor in version
		0.2.2.37.

	\item All physical machines had multiple fingerprints.  The same relays were
		previously characterized by Winter \ea~\cite[\S~5.1]{Winter2016a}.
\end{enumerate}

The remaining two keys belonged to a relay named ``DesasterBlaster,'' whose
origins we could not determine.

\subsection{Shared moduli}
\label{sec:shared-moduli}
In addition to finding shared prime factors, we discovered relays that shared a
\emph{modulus}, giving them the ability to calculate each other's private keys.
Table~\ref{tab:moduli} illustrates these ten relays, clustered into four groups
that each share a modulus.  The table further shows the relays' truncated,
four-byte fingerprint, IP addresses, and RSA exponents.  Note that the Tor
client hard-codes the RSA exponent to 65,537~\cite[\S~0.3]{torspec}, a
recommended value that is resistant to attacks against low public
exponents~\cite[\S~4]{Boneh1999a}.  Any other value indicates non-standard key
generation.  All IP addresses were hosted by OVH, a popular French hosting
provider, and some of the IP addresses hosted two relays, as our color coding
indicates.  Finally, each group shared a four- or five-digit prefix in their
fingerprints.  We believe that a single attacker controlled all these relays
with the intention to manipulate the distributed hash table that powers onion
services~\cite{Biryukov2013a}---the shared fingerprint prefix is an indication.
We suspect that the attackers iterated over exponents to come up with the shared
prefix.  The Tor Project informed us that it discovered and blocked these relays
in August 2014~\cite{tor-priv1} when they first came online.

\begin{table}[t]
	\centering
	\begin{tabular}{l l r}

	Short fingerprint & IP address & Exponent \\
	\midrule

	\texttt{\setlength{\fboxsep}{0pt}%
	\colorbox{highlight1}{\strut 838A}296A} & 188.165.164.163 &
	1,854,629 \\

	\texttt{\setlength{\fboxsep}{0pt}%
	\colorbox{highlight1}{\strut 838A}305F} &
	{\setlength{\fboxsep}{0pt}\colorbox{highlight3}{\strut 188.165.26.13}} &
	718,645 \\

	\texttt{\setlength{\fboxsep}{0pt}%
	\colorbox{highlight1}{\strut 838A}71E2} &
	{\setlength{\fboxsep}{0pt}\colorbox{highlight2}{\strut 178.32.143.175}} &
	220,955 \\

	\midrule

	\texttt{\setlength{\fboxsep}{0pt}%
	\colorbox{highlight1}{\strut 2249E}B42} &
	{\setlength{\fboxsep}{0pt}\colorbox{highlight3}{\strut 188.165.26.13}} &
	4,510,659 \\

	\texttt{\setlength{\fboxsep}{0pt}%
	\colorbox{highlight1}{\strut 2249E}C78} &
	{\setlength{\fboxsep}{0pt}\colorbox{highlight2}{\strut 178.32.143.175}} &
	1,074,365 \\

	\midrule

	\texttt{\setlength{\fboxsep}{0pt}%
	\colorbox{highlight1}{\strut E1EF}A388} & 188.165.3.63 &
	18,177 \\

	\texttt{\setlength{\fboxsep}{0pt}%
	\colorbox{highlight1}{\strut E1EF}8985} &
	{\setlength{\fboxsep}{0pt}\colorbox{highlight4}{\strut 188.165.138.181}} &
	546,019 \\

	\texttt{\setlength{\fboxsep}{0pt}%
	\colorbox{highlight1}{\strut E1EF}9EB8} &
	{\setlength{\fboxsep}{0pt}\colorbox{highlight5}{\strut 5.39.122.66}} &
	73,389 \\

	\midrule

	\texttt{\setlength{\fboxsep}{0pt}%
	\colorbox{highlight1}{\strut 410B}A17E} &
	{\setlength{\fboxsep}{0pt}\colorbox{highlight4}{\strut 188.165.138.181}} &
	1,979,465 \\

	\texttt{\setlength{\fboxsep}{0pt}%
	\colorbox{highlight1}{\strut 410B}B962} &
	{\setlength{\fboxsep}{0pt}\colorbox{highlight5}{\strut 5.39.122.66}} &
	341,785 \\

	\end{tabular}

	\caption{Four groups of relays that have a shared modulus.  All relays
	further share a fingerprint prefix in groups of two or three, presumably to
	manipulate Tor's distributed hash table.}
	\label{tab:moduli}
\end{table}

\subsection{Unusual exponents}
\label{sec:unusual-exponents}
The Tor source code hard-codes the public RSA exponent to 65,537, which is best
practice~\cite[\S~4]{Boneh1999a}.  Being interested in non-standard key
generation, we checked if our data set featured relays with a different
exponent.  Non-standard exponents suggest that a relay was after a specific
fingerprint, to position itself in Tor's hash ring.\footnote{A different
approach to detecting relays that position themselves in the hash ring is to
determine how often they change their fingerprint.~\cite[\S~4.3.3]{Winter2016a}}
To obtain a fingerprint with a given prefix, one can either iterate over the
primes constituting $N$, or over the exponent.  The latter is significantly more
efficient because it is costly to verify if a large number is prime.  Leveraging
this method, the tool Scallion~\cite{scallion} generates vanity onion service
domains by iterating over the service's public exponent.

Among all our 3.7 million keys, 122 possessed an exponent other than
65,537.\footnote{We list all relays in detail in
Appendix~\ref{sec:full-unusual-exponents}.} A single relay had both a
non-standard identity and onion key while all remaining relays only had
non-standard identity keys.  Ten of these relays further have a shared modulus,
which we discuss in Section~\ref{sec:shared-moduli}.  Assuming that these relays
positioned themselves in the hash ring to attack an onion service, we were
interested in learning what onion services they targeted.  It is possible to
identify the victims by compiling a comprehensive list of onion services, and
determining each service's position in the hash ring at the time the malicious
HSDirs were online.  An onion service's position in the hash ring is governed by
the following equations.

\begin{equation}
\begin{split}
\textit{secret-id-part} = \textit{SHA-1}(& \textit{time-period} \mid \\
                                         & \textit{descriptor-cookie} \mid \\
                                         & \textit{replica}) \\
\textit{descriptor-id} =  \textit{SHA-1}(& \textit{permanent-id} \mid \\
                                         & \textit{secret-id-part})
\end{split}
\end{equation}

\textit{Secret-id-part} is comprised of three variables.  \textit{Time-period}
represents the number of days since the Unix epoch; \textit{descriptor-cookie}
is typically unused and hence empty; and \textit{replica} is set to both the
values 0 and 1, resulting in two hashes for \textit{secret-id-part}.  Finally,
the concatenation of both \textit{permanent-id} (the onion service's hashed
public key) and \textit{secret-id-part} is hashed, resulting in
\textit{descriptor-id}, which determines the position in the hash ring.  The
three hidden service directories (for both replicas) whose fingerprint is the
closest to the descriptor in positive direction are the onion service's HSDirs.
This set of six HSDirs changes each day when \textit{time-period} increments.

Knowing the algorithm that determines a service's position in the hash ring, we
then obtained a list of onion services by augmenting the list of the Ahmia
search engine~\cite{ahmia} with services that we discovered via Google searches
and by contacting researchers who have done similar work.  We ended up with a
list of 17,294 onion services, that were online at some point in time.  Next, we
developed a tool that takes as input our list of onion services and the
malicious HSDirs we discovered.\footnote{The tool is available online at
\url{https://github.com/citp/weak-tor-keys/tree/master/code/itos}.}  The tool
then determines all descriptors these onion services ever generated, and checks
if any HSDir would have been close in the hash ring to either of these
descriptors. It is difficult to identify all targeted onion services with
certainty.  First, our list of onion services does not reveal when a service was
online.  Second, a HSDir could be responsible for an onion service simply by
chance, instead of on purpose, resulting in a false positive.  Third, our list
of onion services is not exhaustive, so we are bound to miss potential victims.

Our tool identified the following three onion services for which we have strong
evidence that they were targeted.

\begin{description}
	\item[\texttt{22u75kqyl666joi2.onion}] The service appears to be offline
		today, so we were unable to see for ourselves what it hosted.  According
		to cached index pages we found online, the onion service hosted a
		technology-focused forum in Chinese.  A subset of relays from
		Table~\ref{tab:group2} targeted the onion service on both August 14 and
		15, 2015 by providing nine out of the total of twelve responsible
		HSDirs.

	\item[\texttt{n3q7l52nfpm77vnf.onion}] As of February 2017, the service is
		still online, hosting the ``Marxists Internet Archive,'' an online
		archive of literature.  Figure~\ref{fig:archive} shows a screenshot of
		the service's index page.  A subset of relays from
		Table~\ref{tab:group1} targeted the onion service from November 27 to
		December 4, 2016.  The malicious HSDirs acted inconsistently,
		occasionally targeting only one replica.

	\item[\texttt{silkroadvb5piz3r.onion}] The onion service used to host the
		Silk Road marketplace, whose predominant use was a market for narcotics.
		The service was targeted by a subset of relays from
		Table~\ref{tab:group4}, from May 21 to June 3, 2013.  The HSDirs were
		part of a measurement experiment that resulted in a blog
		post~\cite{OCearbhaill2013a}.
\end{description}

\begin{figure}[t]
	\centering
	\includegraphics[width=\linewidth]{figures/marxists-internet-archive.jpg}
	\caption{A screenshot of the index page of the onion service
		\texttt{n3q7l52nfpm77vnf.onion}, taken on February 13, 2017.}
	\label{fig:archive}
\end{figure}
